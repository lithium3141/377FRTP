\documentclass{article}

\usepackage{amsmath}
\usepackage{amsfonts}
\usepackage{array}
\usepackage{mdwlist}
\usepackage{wasysym}
\usepackage{fancyhdr}
\usepackage{graphicx}

\pagestyle{fancy}
\headheight 35pt
\begin{document}

\lhead{\textbf{CSSE377 \\ Functional Refactoring}}
\rhead{November 2011 \\ Clojure}

\section{Clojure}

Clojure is a dialect of Lisp, developed as a functional companion to Java that targets existing Java Virtual Machines (JVMs) and is therefore compatible with preexisting Java projects. 

\subsection{Language Background}

Clojure first appeared in 2007 as a personal project of Rich Hickey, who explained that he wanted a language which:

\begin{itemize*}
\item was a Lisp dialect
\item supported functional programming paradigms
\item was ``symbiotic'' with an existing, established platform
\item was designed for concurrency
\end{itemize*}

No such language existed to Hickey's satisfaction, and thus Clojure was born. Since inception, Clojure has rapidly expanded to become a complete Lisp dialect, unconstrained by what Hickey calls the ``slow innovation''\cite{1} of both Common Lisp and Scheme since their respective standardizations. Since Clojure is not restricted by the scope of those dialects' standards, it can become a more flexible language, so long as it adheres to the JVM specification in compilation.

Clojure is a compiled language. Once written, Clojure code is converted to JVM bytecode that can be natively interpreted by any JVM, including the Sun standard virtual machines. As compiled, it supports the complete Java specification, including type hinting and inference; it avoids reflection where possible.

Furthermore, as a Lisp dialect, Clojure is highly functional. It contains a lambda calculus core, and extends the paradigm of ``code as data'' to Java maps and vectors. All data types implemented in Clojure are immutable and persistent, making both recursion and concurrency easy and reliable.

Finally, Clojure adheres to Java's polymorphic behavior, providing the ability to subclass and abstract certain parts of written code. It does not provide its own class system, instead preferring to provide many methods that operate on a relatively small number of classes and objects. Hickey asserts that ``inheritance is \textit{not} the only way to do polymorphism''\cite{1}, and attempts to get away from the object-oriented paradigm while still embracing the underlying Java platform.

\subsection{Refactoring in Clojure}

As with most functional languages, Clojure is not subject to the usual set of refactoring techniques available for object-oriented architectures. Instead, Clojure has developed its own set of potential refactorings, nurtured and encouraged by Hickey.

Note: throughout the remainder of this section, code blocks may contain comments prefixed by the traditional C double forward slash; this is not standard Clojure, but is used for readability and conciseness. The actual Clojure comment operator is \verb!(comment ...)!.

\subsubsection{Rename Variable}

One of the simplest refactoring techniques available in Clojure is a simple rename. While relatively trivial to do, this is also a fairly powerful technique - Hickey has gone so far as to call it ``the number one refactoring''\cite{2}. The simplicity of the technique comes from the static typing of names within any program. For example, when attempting to refactor a variable \verb!bar! within the \verb!foo! namespace (written as \verb!foo/bar!), it is trivial for a good IDE or compiler to determine what instances of \verb!bar! are syntactically equivalent to the target \verb!bar! by checking the variables' scope and namespaces. Hence, given the Clojure snippet:

\begin{verbatim}
(def bar 1)

(defn foo[]
    (let [bar 2]
        (println bar)))
\end{verbatim}

Renaming the inner variable \verb!bar! would resolve to the local instance defined within the \verb!let! statement (and hence apply the rename refactoring to that \verb!let!), while renaming the instance of \verb!bar! defined in the \verb!def! statement would have no other effects (since there are no inner instances of \verb!bar!) that resolve to that definition. The results of these two refactorings are shown as follows:

\begin{tabular}{m{2in} m{2in}} \\
\begin{verbatim}
// Rename def'd bar
(def baz 1)

(defn foo[]
    (let [bar 2]
        (println bar)))
\end{verbatim}
&
\begin{verbatim}
// Rename let'd bar
(def bar 1)

(defn foo[]
    (let [baz 2]
        (println baz)))
\end{verbatim}
\end{tabular}

The advantage of this refactoring technique lies primarily in its potential to enhance code readability and maintainability - a developer who can make sense of variable naming schemes and understanding the meaning underlying a piece of code is much more likely to be productive and make workable changes than one who has no idea what's going on in a code snippet. Furthermore, this technique is especially powerful because of the simplicity of its application - a sufficiently ``intelligent'' editing environment can easily handle this sort of change, and in fact several projects exist on GitHub that handle precisely this refactoring (most notably \verb!clojure-refactoring! by Tom Crayford \cite{3}).

\subsubsection{Extract Expression}

A refactoring that has a strong similarity to a well-known object-oriented technique, extracting an expression also has powerful implications for many Clojure programs. Many times, developers will find a common task being performed repeatedly, and would benefit strongly from taking that task or expression and extracting it each place it is used, placing it into a function instead.

Consider, as a trivial example, a reimplementation of the Java method \verb!modPow!, which takes as arguments three integers $a, b, c$ and returns the result of the expression $a^{b} \; mod \; c$. (This method has common use in cryptographic and modular arithmetic fields.) We could write, for example, an encryption-decryption algorithm for RSA which would take a message $m$ and a public key $(n, e)$, encrypt it, decrypt it with a private key $d$, and assert that the decrypted text is equivalent to the original method (e.g. for implementation testing). Without refactoring, this method might look like:

\begin{verbatim}
// Encrypt and decrypt a message using RSA
// Args: message m, public key (n, e), private key d
// Returns true if encrypted-decrypted message matches original
(defn encryptDecrypt [m n e d]
    (let [c (mod (pow m e) n)]
        (let [m2 (mod (pow c d) n)]
            (= m m2))))
\end{verbatim}

We note that the pattern \verb!(mod (pow x y) z)! appears multiple times, and thus we may extract that expression, simplifying the \verb!let! statement and making the functionality it provides available to other users:

\begin{verbatim}
// Exponentiate and mod the given numbers
// Args: x y z; returns ((x ^ y) mod z)
(defn modPow [x y z]
    (mod (pow x y) z))

// Encrypt and decrypt a message using RSA
// Args: message m, public key (n, e), private key d
// Returns true if encrypted-decrypted message matches original
(defn encryptDecrypt [m n e d]
    (let [c (modPow m e n)]
        (let [m2 (modPow c d n)]
            (= m m2))))
\end{verbatim}

\subsubsection{Flatten Reduce}

This technique is one of the first not to have a clear object-oriented counterpart. It deals strongly with the \verb!reduce! function, which takes a sequence (the Clojure term for a collection of values) and applies a function repeatedly until the sequence is reduced to a single variable. This function finds heavy use in so-called \textit{map-reduce} applications, where sets of data are broken apart, transformed with the \verb!map! function, and merged back into single sets. Due to the chunked nature of the task, map-reduce programs are commonly written in highly concurrent functional languages; as such, good refactoring techniques can greatly improve a common application of Clojure and other functional languages.

To illustrate this particular refactoring technique, consider the task of summing numbers contained in a three-dimensional array (represented in Clojure by nested symbol lists). While this can be done recursively, for the purposes of flattening reduces we write the original scenario using the \verb!reduce! function:

\begin{verbatim}
// For convenience, predefine the dataset (a 3D integer array)
(def xs '(((1 2) (3 4)) ((5 6) (7 8))))

// Provide the sum of all integers in the set
(reduce (fn [acc x]
    (reduce (fn [acc y]
        (reduce (fn [acc z] (+ acc z)) acc y)) acc x)) 0 xs)
\end{verbatim}

The repeated explicit application of \verb!reduce! is what we seek to eliminate with this refactoring. In Clojure, the \verb!for! keyword allows developers to apply certain operations over nested sets of objects by flattening those sets; we can apply a \verb!for! operation on the \verb!reduce! to improve the clarity and conciseness of this code block:

\begin{verbatim}
// For convenience, predefine the dataset (a 3D integer array)
(def xs '(((1 2) (3 4)) ((5 6) (7 8))))

// Apply the reduce for each of the three dimensions
(reduce (fn [a b] (+ a b)) 0 (for [x xs, y x, z y] z))
\end{verbatim}

Whether this refactoring truly improves clarity is a subject of some debate; the syntax of the \verb!for! operator is not immediately obvious, especially if the underlying data structure is not understood. In this example, the \verb!for! expands \verb!xs! to a set of objects represented individually as \verb!x!, then does the same for each \verb!y! in \verb!x! and \verb!z! in \verb!y!. It returns a single flattened sequence of integers that originally belonged to separate sublists in the three-dimensional \verb!xs!; at that point, the \verb!reduce! may be applied over the single sequence. Note that the structure of the original dataset can be ``ignored'' in this manner since it is being reduced away regardless; operations where nesting level matters, or where additional information must be retained about the original data structure, flattening reduces is not always possible.

\subsubsection{Compose Function}

Functional languages tend to borrow somewhat heavily from the more formally mathematical side of computer science, and Clojure is no exception -- its core in lambda calculus and pure nature make it a strong candidate for mathematical applications. As such, some refactoring techniques are derived from mathematical operations, such as the composition of functions.

Consider a basic Clojure implementation of the ROT-13 encryption algorithm on a sequence of integers representing ASCII values of capital letters. In this algorithm, there are four basic building blocks:

\begin{itemize*}
\item Subtracting 65 to find the 0-indexed position within the alphabet of the given value
\item Adding 13 for the rotation
\item Taking the modulus (base 26) of the new value
\item Adding 65 to return to the domain of ASCII values
\end{itemize*}

In an overly verbose implementation, each of these might be separated, as such:

\begin{verbatim}
(defn asciiToZeroIndex [x] (- x 65))
(defn zeroIndexToAscii [x] (+ x 65))
(defn rotate [x] (+ x 13))
(defn alphabetMod [x] (mod x 26))

// The function might then be applied as
(map zeroIndexToAscii
    (map alphabetMod
        (map rotate
            (map asciiToZeroIndex 
                '(72 69 76 76 79 87 79 82 76 68)))))
\end{verbatim}

While very compartmentalized and almost painfully obvious as to the purpose of individual functions, the application of this ROT-13 implementation is somewhat lacking in clarity and quite verbose. In addition, there is rarely a call for any of those functions outside the context of the others -- when does a programmer turn to a function called \verb!alphabetMod! instead of simply performing the \verb!mod!? As such, we may compose these four functions together to come up with a single clean ROT-13 function:

\begin{verbatim}
(defn rot13 [x] (+ (mod (+ (- x 65) 13) 26) 65))

// Apply function
(map rot13 '(72 69 76 76 79 87 79 82 76 68))
\end{verbatim}

\subsection{Conclusion}

Clojure occupies an interesting space in the field of programming languages as one that is undeniably functional, but at the same time integrates with the very heavily object-oriented Java. As such, its refactoring techniques comprise a number of common methods from known OO refactoring strategies, while also introducing some new ideas that are much more functional (and even mathematical at times). Notably, the more object-oriented refactorings available tend to be simpler - renaming variables, extracting functions, and so forth - while the more functional strategies reach much greater complexities (e.g. flattening a \verb!reduce! across a nested data structure).

\begin{thebibliography}{9}
\bibitem{1} Hickey, Richard. "Clojure: Rationale." Clojure - Rationale. Web. 28 Oct. 2010. \verb!<http://clojure.org/rationale>!.
\bibitem{2} Hickey, Richard. "Re: Refactoring? - Richard Hickey - com.googlegroups.clojure - MarkMail." Mailing List. com.googlegroups.clojure. Google Groups, 23 July 2008. Web. 31 Oct. 2010. \verb!<http://markmail.org/message/btbnpxvuyob5xrig>!.
\bibitem{3} Crayford, Tom. tcrayford's Clojure-refactoring at Master - GitHub. Clojure-refactoring. 23 Jan. 2010. Web. 31 Oct. 2010. \verb!<http://github.com/tcrayford/clojure-refactoring>!.
\end{thebibliography}

\end{document}

\documentclass{article}

\usepackage{amsmath}
\usepackage{amsfonts}
\usepackage{array}
\usepackage{mdwlist}
\usepackage{wasysym}
\usepackage{fancyhdr}
\usepackage{graphicx}

\pagestyle{fancy}
\headheight 35pt
\begin{document}

\lhead{\textbf{CSSE377 \\ Functional Refactoring}}
\rhead{November 2011 \\ YOUR LANGUAGE}

\section{LANGUAGE}

You are responsible for at least \textbf{five pages} of paper on your language with default margins, single-spaced. I'd like to see something at least close to final by \textbf{Monday, November 1}. Language assignments:

\begin{itemize*}
\item Tim - Clojure
\item Stokes - Erlang
\item Pete - Scala
\item Tom - Haskell
\item David - Scheme
\end{itemize*}

\subsection{Language Background}

Write 1--1.5 pages on the history of your language, its common uses today, and any particular idiosyncrasies or quirks your language possesses. Assume your reader is technical, but has no background in your language specifically; relate it to other common languages. Do not reference any other functional languages we're using.

\subsection{Refactoring in LANGUAGE}

Write 3--3.5 pages about refactoring in your particular language. Be sure to include code snippets and follow through the refactoring process of at least one decently-sized chunk of code. You can use

\begin{verbatim}
// code blocks
\end{verbatim}

as well as \verb!inline verbatim! blocks to format code. Don't forget about $math \; markup$ if you're dealing with anything mathy like algorithm analysis or functional composition. Also be sure to cite your sources! (This may require you to compile the paper twice to make the citations work\cite{Item1}.)

\subsection{Conclusion}

Wrap up your section with a 0.5--1 page conclusion that recaps refactoring in your language and discusses any emerging concepts or practices in your language.

\begin{thebibliography}{9}
\bibitem{Item1} Item One
\end{thebibliography}

\end{document}
